% \documentclass{article}

% \usepackage{amsmath}
% \usepackage{amssymb}
% \usepackage{listings}
% \usepackage{graphicx}

% \title{Numerical Analysis Using Finite Difference Method}
% \author{Your Name}
% \date{}

% \begin{document}

% \maketitle

% \section{Introduction}

% Numerical analysis is a branch of mathematics that deals with the development and implementation of algorithms for solving mathematical problems. One common approach to numerical analysis is the finite difference method, which involves approximating derivatives of a function by evaluating it at a set of discrete points.

% The finite difference method can be used to solve a wide range of problems, including differential equations, optimization problems, and interpolation. In this article, we will focus on its application to the solution of differential equations.

% \section{Finite Difference Method}

% Consider the following second-order differential equation:

% \begin{equation}
% y''(x) = f(x)
% \end{equation}

% where $y(x)$ is the unknown function, and $f(x)$ is a given function. To solve this equation using the finite difference method, we first need to discretize the domain of $x$ into a set of $N$ equally spaced points, with a spacing of $h$. Let $x_i = ih$, for $i=0,1,\dots,N$.

% We can then approximate the second derivative of $y(x)$ at $x_i$ using a finite difference formula:

% \begin{equation}
% y''(x_i) \approx \frac{y(x_{i+1}) - 2y(x_i) + y(x_{i-1})}{h^2}
% \end{equation}

% Using this approximation, we can rewrite the differential equation (1) as a system of linear equations:

% \begin{equation}
% \frac{1}{h^2} \begin{bmatrix}
% 1 & 0 & 0 & \dots & 0 \\
% 1 & -2 & 1 & \dots & 0 \\
% 0 & 1 & -2 & \dots & 0 \\
% \vdots & \vdots & \vdots & \ddots & \vdots \\
% 0 & 0 & \dots & 1 & -2 \\
% 0 & 0 & \dots & 0 & 1
% \end{bmatrix} \begin{bmatrix}
% y_0 \\
% y_1 \\
% y_2 \\
% \vdots \\
% y_{N-1} \\
% y_N
% \end{bmatrix} = \begin{bmatrix}
% f(x_0) \\
% f(x_1) \\
% f(x_2) \\
% \vdots \\
% f(x_{N-1}) \\
% f(x_N)
% \end{bmatrix}
% \end{equation}

% where $y_i \approx y(x_i)$.

% Solving this system of linear equations gives us an approximation to the solution of the differential equation (1) at the discrete points $x_i$.


% \begin{lstlisting}[language=Python]
%   import numpy as np
%   import matplotlib.pyplot as plt
%   from matplotlib.animation import FuncAnimation
  
%   # Define functions
%   def diff2_central(f, x, h=0.1):
%       return (f(x+h)-2*f(x)+f(x-h))/h**2
  
%   def f(xx):
%       C = (11-81/12)/3
%       return xx**4/12 - xx**2 + C*xx
  
%   def fn(xx):
%       return xx**2 - 2
  
%   # Set up the graph
%   fig, ax = plt.subplots()
%   ax.set_xlabel('x')
%   ax.set_ylabel('y')
%   ax.set_xlim(0, 3)
%   ax.set_ylim(-1, 1)
  
%   # Plot the initial function f(x)
%   x = np.arange(0, 3, 0.01)
%   y = f(x)
%   ax.plot(x, y, color='red')
  
%   # Define variables and initial conditions
%   z = 3
%   x = 0
%   dx = 0.01
%   h = 0.2
%   xp = np.arange(0, 3+h, h)
%   m = f(z)/z
%   fp = m*xp
%   fp2 = m*xp
  
%   # Plot the initial function fp(x)
%   line, = ax.plot(xp, fp, color='blue', marker='o')
  
%   # Create an array of colors
%   colors = ['orange', 'green', 'purple', 'brown', 'teal', 'maroon', 'navy']
  
%   # Define update function for animation
%   def update(n):
%       global fp, fp2
%       fdata = []
%       for i in range(1, len(xp)-1):
%           fp2[i] = 0.5*(fp[i+1]+fp[i-1]-h**2*(fn(xp[i])))
  
%       for i in range(len(xp)):
%           fp[i] = fp2[i]
%           fdata.append([xp[i], fp[i]])
  
%       # Update the line data with the new values
%       line.set_data(xp, fp)
  
%       # Set line color based on the color array and current iteration
%       color_index = n // 50 % len(colors)
%       line.set_color(colors[color_index])
  
%       return [line]
  
%   # Create animation object
%   anim = FuncAnimation(fig, update, frames=range(N), interval=50)
  
%   plt.show()
%   \end{lstlisting}

% \section{Conclusion}

% The finite difference method is a powerful technique for solving a wide range of problems in numerical analysis. In this article, we have seen how it can be applied to the solution of differential equations. By discretizing the domain of a function into a set of discrete points, we can use finite difference formulas to approximate derivatives and rewrite differential equations as systems of linear equations. Solving these systems of linear equations gives us an approximation to the solution of the original problem.

% \end{document}


\documentclass{article}

\usepackage{amsmath}
\usepackage{amssymb}
\usepackage{listings}
\usepackage{graphicx}
\usepackage{algorithm}
\usepackage{algpseudocode}

\title{Numerical Analysis Using Finite Difference Method}
\author{Eric Kazungu \and Samuel Wainaina \and Keith Kareithi}

\date{27 March 2023}

\begin{document}

\maketitle
\section*{Introduction}
This method is used to find the solution to differential equations including partial differential equations. This means that given a function's derivatives, we can obtain the original function numerically using this method. This method actually stems from the very first idea we learned when dealing with differentiation where we use it to find the derivative of a function. Given a sufficiently smooth function $f(x)$, the derivative of $f$ is defined as
\begin{equation*}
f'(x)=\frac{f(x+h)-f(x)}{h}.
\end{equation*}
For instance, if the point is 2 in the curve we can pick point $f(x+h)=f(3)$. In this, we can calculate the gradient as $y_x$ where a change in $y$ is $f(3)-f(2)$ and change in $x$ is $h$. In this case, the gradient from this calculation is a very bad estimate, however, this estimate gets better as $h$ gets smaller and smaller. This is the basis for the finite difference method.

\textbf{The three types of difference methods}

With this in mind, we can also say that there are two other possibilities. The first one we just went through in the introduction is called the forward difference method. The others are the following:

\begin{enumerate}
\item \textbf{Backward difference method}: In this case, we use the value behind the chosen point to find the value of $f'(x)$
\begin{equation*}
f'(x)=\frac{f(x)-f(x-h)}{h}
\end{equation*}
\item \textbf{Central difference method}: This is the interesting one amongst the three because it can easily help us find the second derivative of $f(x)$. In this case, we use the 2 points on the adjacent sides of our chosen point $f(x)$ to find $f'(x)$
\begin{equation*}
    f'(x)=\frac{f(x+h)-f(x-h)}{2h}
\end{equation*}

Second derivative using central difference method: In order to do this, we need the point between $f(x+h)$ and $f(x)$ as well as the point between $f(x-h)$ and $f(x)$ and find the gradient for them using the central difference method, i.e.,

\begin{align*}
    f'(x+\frac{1}{2}h)&=\frac{f(x+h)-f(x)}{h} \\
    f'(x-\frac{1}{2}h)&=\frac{f(x)-f(x-h)}{h}
\end{align*}

Then apply the method again to these 2 gradients to find $f''(x)$:
\begin{align*}
    f''(x)&=\frac{f'(x+\frac{1}{2}h)-f'(x-\frac{1}{2}h)}{h} \\
    &=\frac{1}{h}\Big[\frac{f(x+h)-f(x)}{h}-\frac{f(x)-f(x-h)}{h}\Big] \\
    &=\frac{1}{h^2}\Big[f(x+h)-f(x)-f(x)+f(x-h)\Big] \\
    &=\frac{1}{h^2}\Big[f(x+h)+f(x-h)-2f(x)\Big]
\end{align*}

In this form, we have a very powerful equation that can help us find the second derivative of a function.
\end{enumerate}

Integration using central difference method: By simply rearranging these equations to make $f(x)$ the subject of the formula, we can derive the original function numerically using the derivatives. This means that we can use the central difference method to find the solution to a differential equation.


\begin{algorithm}
  \caption{Finite Difference Method}
  \begin{algorithmic}[1]
  \State Set the boundaries and conditions which are the maximum iterations($N$), the step size($h$), the lower limit($LL$), and the upper limit($UL$).
  \State Define the function $f(x)$.
  \State Define the second-order derivative for the function, denoted as $f''(x)$.
  \State Get the values of $f(UL)$ and $f(LL)$ and use them to calculate the gradient $\frac{f(UL)-f(LL)}{UL-LL}$.
  \State Create a list of elements within the range of $[LL, UL]$ in intervals of $h$.
  \State For each element $x$ in the list obtained in step 5, determine the value of $y$ by multiplying $x$ with the gradient obtained in step 4.
  \State For each point $(x,y)$ obtained in step 6, use the formula $f(x)=f(x+h)+f(x-h)-f''(x)h^2$ substituting $f''(x)$ with the defined differential equation to calculate the approximate values of $f(x)$ for the solution to the differential equation.
  \State Plot these approximate values alongside the graph of the exact function $f(x)$ to validate the accuracy of the code.
  \end{algorithmic}
  \end{algorithm}

\end{document}